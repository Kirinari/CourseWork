\documentclass{article}
\usepackage[russian]{babel, amsfonts,longtable, amssymb, amsmath}
\usepackage[left=25mm, top=20mm, right=10mm, bottom=15mm, nohead, nofoot]{geometry}
\linespread{1.3}
\usepackage{graphicx}
\graphicspath{{H:/Desktop/КУРСОВАЯ/}}
\begin{document}
	\begin{titlepage}
		\begin{center}
			{\small \sc Московский государственный университет \\имени М.~В.~Ломоносова\\
				Факультет вычислительной математики и кибернетики\\}
			\vfill
			{\Large \sc Курсовая работа}\\
			~\\
			{\large \bf <<...>>}\\ 
			~\\
		\end{center}
		\begin{flushright}
			\vfill {Выполнил:\\
				студент 311 группы\\
				Сомов~А.~А.\\
				~\\
				Научный руководитель:\\
				Давидсон~М.~Р.}
		\end{flushright}
		\begin{center}
			\vfill
			{\small Москва\\2021}
		\end{center}
	\end{titlepage}
	{\Large \bf Обзор данных}
	
	{~}
	
	Мы имеем портфель из следующих производных финансовых инструментов:
	
	\begin{itemize}
		\item RI - маржируемый опцион колл на фьючерсный контракт на индекс РТС
		\item Si - маржируемый опцион колл на фьючерсный контракт на курс доллар США - российский рубль
		\item GZ - маржируемый опцион колл на фьючерсный контракт на обыкновенные акции ПАО «Газпром»
		\item SR - Маржируемый опцион колл на фьючерсный контракт на обыкновенные акции ПАО Сбербанк
	\end{itemize}
	
	Дополнительная информация по рассматриваемым опционам представлена на {\slshape рис.1}:
	
	{~}
	\begin{center}
		\begin{tabular}{| l | r | c | r |}
			\hline
			Инструмент & страйк & дата экспирации & полный код контракта на ММВБ  \\ \hline
			RI & 115000 & 17.06.2021 & RTS-6.21M170621CA115000 \\ \hline
			Si & 66000 & 17.06.2021 & Si-6.21M170621CA66000 \\ \hline
			GZ & 18500 & 16.06.2021 & GAZR-6.21M160621CA18500 \\ \hline
			SR & 24500 & 16.06.2021 & SBRF-6.21M160621CA24500 \\ \hline
			
		\end{tabular}
		
		{~}
		
		{\slshape рис.1}
	\end{center}
	
	
	На { \slshape рис.2} представлены графики цен на указанные инструменты в период с 19.08.2020
	по 22.01.2021.
	
	На { \slshape рис.3} графики цен на базовые активы соответствующих опционов за тот же период.
	
	На {\slshape рис.4} для наглядности приведены графики цен всех опционов и их базовых активов
	
	\begin{center}
		\begin{tabular}{c c}
			\includegraphics[scale = 0.5]{RI.png} & \includegraphics[scale = 0.5]{GZ.png} \\
			\includegraphics[scale = 0.5]{Si.png} & \includegraphics[scale = 0.5]{SR.png} \\
		\end{tabular}
		
		{\slshape рис.2}
		
		\begin{tabular}{c c}
			\includegraphics[scale = 0.5]{RIM1.png} & \includegraphics[scale = 0.5]{GZM1.png} \\
			\includegraphics[scale = 0.5]{SiM1.png} & \includegraphics[scale = 0.5]{SRM1.png} \\
		\end{tabular}
		
		{\slshape рис.3}
	\end{center}
	
	\begin{center}
		
		\includegraphics[scale = 0.5]{all.png}
		
		{\slshape рис.4}
		
	\end{center}
	\newpage
	{\Large \bf Постановка задачи}
	
	{~}
	
	Рассмотрим промежуток времени $H$. В момент $t = 0$ мы имеем начальный капитал $W_0$. Мы можем вкладывать его в моменты времени $ t = 1, \ldots, H-1$ в инструменты $x_i,~i = \{1, 2, 3, 4\}$, где 
	\begin{itemize}
		\item $x_1$- соответствует RI
		\item $x_2$ - Si
		\item $x_3$ - GZ
		\item $x_4$ - SR
	\end{itemize}
	
	В момент времени $t = H$ наш капитал должен удовлетворять требованиям $L$, то есть $W_H \ge L.$\\
	Введём бинарное дерево $D$ с множеством вершин $N,~ |N| = k$. Будем обозначать его вершины $n_i, ~n_i \in N, ~i~=~1,\ldots,k.$ Обозначим $S,~S \subset N$ множество листовых вершин дерева D, а $T,~T \subset N$ - множество его внутренних вершин. Каждая вершина соответствует какому-то моменту времени $t$ таким образом, что корень дерева $n_0$ соответствует моменту $t = 0,~n_1$ и $n_2$ соответствуют $t = 1$ и так далее с учётом того, что дерево D бинарное.
	
	{~}
	
	Например, для $k = 15$ имеем {\slshape (рис.5)}\\
	\begin{center}
		\includegraphics[scale = 0.5]{graph.png}
		
		{\slshape рис.5}
	\end{center}
	
	\newpage
	
	Введём ещё несколько необходимых обозначений. \\
	\begin{itemize}
		\item $a(n)$ - родитель вершины $n$
		\item $x_i^n \ge 0$ - количество инструмента $x_i$ в вершине $n$ (после покупки или продажи)
		\item $z_i^n \ge 0$ - размер покупки инструмента $x_i$ в вершине $n$
		\item $y_i^n \ge 0$ - размер продажи инструмента $x_i$ в вершине $n$
		\item $W^s \ge 0$ - размер капитала в листовой вершине $s \in S$
		\item $\pi^s$ - вероятность попасть в листовую вершину $s \in S$ из корневой вершины $n_0$. Эта вероятность рассчитывается по всем путям из в $n_0$ в $s$
		\item $L^n$ - обязательства, которым должен удовлетворять капитал в вершине $n$, то есть $W^n \ge L^n$
		\item $c$ - стоимость транзакции (покупки или продажи какого-либо инструмента) в процентах
		\item $h_i^{n_0}$ - начальное количество инструмента $x_i$ в корневой вершине $n_0$
		\item $P_i^n$ - цена инструмента $x_i$ в вершине $n$
		\item $u(W)$ - функция полезности
		
		Теперь мы можем поставить задачу оптимизации: 
		
		\begin{center}
			$$max~\sum\limits_{s \in S} \pi^su(W^s)$$
			$$x_i^{n_0}
		\end{center}
	\end{itemize}
	
	
	
	
	
	
	
	
	
	
	
	
	
	
	
	
	
	
	
	
	
\end{document}